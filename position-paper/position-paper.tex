\documentclass{article}

    \usepackage[T1]{fontenc}
    \usepackage{ucs}
    \usepackage[utf8]{inputenc}

    \usepackage{acronym} % \ac[p], \acl[p], \acs[p], \acf[p]
    \usepackage{authblk} % \affil
    \usepackage[margin=1in]{geometry}
    \usepackage{subcaption}
    \captionsetup{subrefformat=parens}

    \usepackage{graphicx}
    \usepackage{color}
    \AtBeginDocument{
    \definecolor{pdfurlcolor}{rgb}{0,0,0}
    \definecolor{pdfcitecolor}{rgb}{0,0,0}
    \definecolor{pdflinkcolor}{rgb}{0,0,0}
    \definecolor{light}{gray}{.85}
    \definecolor{vlight}{gray}{.95}
    \definecolor{darkgreen}{rgb}{0.0, 0.2, 0.13}
    }
    \usepackage[colorlinks=true,
                citecolor=pdfcitecolor,
                urlcolor=pdfurlcolor,
                linkcolor=pdflinkcolor,
                pdfborder={0 0 0}
                ]{hyperref}
    \usepackage{url}
    \urlstyle{rm}
    \usepackage{paralist}
    \usepackage{booktabs}

    \usepackage[numbers]{natbib}

    \newcommand{\email}[1]{\href{mailto:#1}{#1}}


    % Acronyms
    % --------
    \acrodef{CRDT}[CRDT]{Conflict-free Replicated Data Type}
    \acrodefplural{CRDT}[CRDTs]{Conflict-free Replicated Data Types}

    % Meta-Data
    % ---------
    \title{Efficient (re)naming in \acp{CRDT}}
    \author{Matthieu Nicolas \\ \email{matthieu.nicolas@inria.fr}}
    \affil{Inria, F-54600,
      Université de Lorraine, LORIA, F-54506,
      CNRS, F-54506}
    \date{}

    \hypersetup{pdftitle={Efficient (re)naming in CRDT},
      pdfsubject={},
      pdfkeywords={Data replication, CRDT},
      pdfauthor={M. Nicolas, O. Perrin, G. Oster}}

    \pagestyle{empty}

    \begin{document}

    \maketitle

    \begin{itemize}
        \item Abstract (maximum of 100 words).
        \item Clear statement of the identified research problem(s) and the context in which the problem(s) will be addressed.
        \begin{itemize}
            \item Distributed applications need to make a trade-off between availability and consistency
            \item To ensure high availability, adopt the optimistic replication model paired with the eventual consistency model to replicate data
            \item A conflict resolution mechanism is needed to solve concurrent updates
            \item The \ac{CRDT} approach
            \item To converge, some \acp{CRDT} attach identifiers to each each element of the data structure
            \item Identifiers' size grows over time, decreasing application's performance
            \item \acp{CRDT} representing replicable sequences are some of the ones suffering the identifiers' growth issue
        \end{itemize}
        \item Summary (with appropriate references) of the state-of-the-art related to the identified problem(s) along with a clear restatement of the “gap” relative to the research problem(s).
        \begin{itemize}
            \item A sequence represents a number of ordered values
            \item Can insert or remove an element at a position
            \item Several \acp{CRDT} allow to replicate sequences : Treedoc, Logoot, LogootSplit
            \item To ensure convergence, attach an identifier to each inserted element
            \item Identifiers allow to identify an element and order them
            \item Thus identifiers have to comply to several constraints : globally unique, totally ordered and belong to dense set
            \item Because of these constraints, grow quicker
            \item In Core and Nebula, authors propose mechanism to rename identifiers to limit growing rate
            \item Proposed mechanism requires consensus and to determine a "core" of stable and durable nodes
            \item Want to address this issue in a fully distributed manner, without any kind of super-peers
        \end{itemize}
        \item Statement explaining the approach and results, according to PhD stage: state the intended approach including a summary of work accomplished to date (if any).
        \begin{itemize}
            \item Defined properties that renaming functions need to comply to (commutative with insert/delete, a rename can be undone, do not shuffle the order)
            \item Designed corresponding functions
            \item Designed a new version of LogootSplit integrating them
        \end{itemize}
        \item Description of evaluation : describe the evaluation plan including intended metrics (quantitative and/or qualitative).
        \begin{itemize}
            \item Implementing this new \ac{CRDT} into MUTE
            \item Testing it in every scenario we were able to think of
            \item Need to prove the correctness of these renaming functions
            \item Taking a look to model checking and automatic theorem provers
            \item In parallel, designing a benchmark for collaborative editing tools (NOTE: mauvais terme ?)
            \item Goal is be able to replay a collaborative editing session from its logs using different conflicts resolution mechanisms
            \item Intend to ensure that the implementation is correct
            \item Intend to mesure the performances (memory, cpu usage and bandwith usage, time to integrate an update, time required to converge)
        \end{itemize}
        \item Conclusion: that includes a statement of the real or potential impact(s) of solving the identified research problem(s). Finishers especially should include a brief conjecture about future work that builds on the PhD research.
        \item Acknowledgements that properly recognize others’ contributions to the work (supervisor(s), other graduate students, funding sources, etc.).
    \end{itemize}

\end{document}
